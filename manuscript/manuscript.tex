\documentclass[review]{elsarticle}

\usepackage{lineno,hyperref}
\modulolinenumbers[5]

\journal{Journal of Archaeological Science}

%%%%%%%%%%%%%%%%%%%%%%%
%% Elsevier bibliography styles
%%%%%%%%%%%%%%%%%%%%%%%
%% To change the style, put a % in front of the second line of the current style and
%% remove the % from the second line of the style you would like to use.
%%%%%%%%%%%%%%%%%%%%%%%

%% Numbered
%\bibliographystyle{model1-num-names}

%% Numbered without titles
%\bibliographystyle{model1a-num-names}

%% Harvard
\bibliographystyle{model2-names.bst}\biboptions{authoryear}

%% Vancouver numbered
%\usepackage{numcompress}\bibliographystyle{model3-num-names}

%% Vancouver name/year
%\usepackage{numcompress}\bibliographystyle{model4-names}\biboptions{authoryear}

%% APA style
%\bibliographystyle{model5-names}\biboptions{authoryear}

%% AMA style
%\usepackage{numcompress}\bibliographystyle{model6-num-names}

%% `Elsevier LaTeX' style
%\bibliographystyle{elsarticle-num}
%%%%%%%%%%%%%%%%%%%%%%%

\begin{document}

\begin{frontmatter}

\title{Combining Bayesian Chronological Models with the Aoristic/Monte-Carlo approach: a case study using a revised Chronology of J\={o}mon Pottery Phases in Japan}



%% or include affiliations in footnotes:
\author[address1]{Enrico R. Crema}
\cortext[mycorrespondingauthor]{Corresponding author}
\ead{erc62@cam.ac.uk}

\author[address2]{Ken'ichi Kobayashi}


\address[address1]{Department of Archaeology, University of Cambridge, Downing Street, CB2 3ER  Cambridge, UK}
\address[address2]{Chuo University, Tokyo, Japan}




\begin{abstract}
Key Highlights
* A Bayesian chronological model of J\=omon Pottery Phases is presented.
* A new approach to Aoristic/Monte-Carlo analysis is proposed
* The absolute chronology of changes in the frequency of J\=omon residential units in Central is presented. 
\end{abstract}


\begin{keyword}
Bayesian Statistics\sep Monte-Carlo Simulation \sep Aoristic Analysis \sep J\=omon Chronology \sep Prehistoric Demography
\MSC[2010] 00-01\sep  99-00
\end{keyword}

\end{frontmatter}

\linenumbers

\section{Introduction}
* set the question on big data, use of relative chronology
* briefly review the use of SPD and its limitations
* briefly review the use of relative chronology via aoristic analysis, and review levels of uncertainty
* highlight lack of approach considering the impact of uncertainty in the relative chronology per se, tendency to use marked intervals
* missed opportunity and big data, highlight relevance of Japanese archaeology
* objective of the paper: new chronology of Jomon pottery phases, new method for integrating temporal uncertainity in aoristic/MC routine, comparison of SPD and residential units counts in SW Kanto using legacy data
* summarise paper structure
* Figure 1 highlight the limitation of Aoristic Sum, and limitation of time-spans without chronological uncertainty

\subsection{Case Study: J\={o}mon Chronology and Demography}
* review of Jomon Pottery Phases
* review of Jomon Demograhy
* review of attempts of absolute chronology


\section{Materials}
* Mention Sampling Criteria
* Number of samples, definition of pottery phases, reference to shinchihei-hennen 
* Suzuki's data set summary

\section{Methods}
\subsection{Bayesian Chronological Modelling}
* brief discussion on the choice of the statistical model, use gaussian with an-priori justification? (reference manning, and also cult evo literature)
* brief mention on appendix with a trapezoid distribution alternative and sensitivity analysis
* mention data preparation, R\_Combine, and outlier analysis
* mention exclusion based on agreement index

\subsection{Aoristic Monte-Carlo Simulation using Bayesian Posteriors}
* briefly review assumpsions of aoristic analysis
* number of simulations etc.


\section{Results}
\subsection{Chronological Model}
* Figure 2 with the chronological model
\subsection{Demographic Models}
* Figure 3a and 3b: pithouse counts, and rates of change between sequential blocks.  


\section{Discussion and Conclusion}
* compare, imamura, crema 2012, and this paper (fig.4a)
* comparison also with Crema et al 2016, highlighting broad similarity (possibly add a figure 4b?)
* stress the point about how with extreme time-span uncertainty the impact of discrete phase boundary gives the impression of an abrupt change, which is unrealistic.
* highlight the fact that the plot displays what was the dynamic at a particular point in time. In other words this does not portray whether an event occurred. For example, the decline in the number of pit-houses recorded between Early and Middle Jomon can be observed in different simulation runs, but because there is a high uncertainty on when this transition took place the composite plot shown in fig.4 does not show a clear pattern. 
* brief summary of results
* challenges and future perspectives, especially on model selection
* the relevance of big data and awareness, and bla bla

\section*{References}

\bibliography{mybibfile}

\end{document}